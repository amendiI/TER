\documentclass[a4paper,10pt]{article}
\usepackage[utf8]{inputenc}
\usepackage[T1]{fontenc}
\usepackage[french]{babel}
\usepackage{amsmath}
\usepackage{hyperref}
\usepackage{amsfonts}

\begin{document}
\title{Rapport TER}
\author{Raphaël Muller,Billel Guerfa}
\maketitle

\part{Introduction}




\part{théorie}
\section{Processus de décision de Markov}
\subsection{Processus de Markov}
Le processus de décision de Markov décrit formellement un environnement pour l' apprentissage par renforcement. Cet environnement est totalement observable et dénombrable.
Les chaines vérifient toutes la propriété de Markov. Un état $S_t$ vérifie les propriété de Markov si et seulement si:
\[\mathbb{P}[S_{t+1}|S_t]=\mathbb{P}[S_{t+1}|S_1,....,S_t]\]
Cela signifie que l'état $S_t$ a  une propriété d'absence de mémoire, c'est a dire que l'état actuel contient toutes les informations importantes des états précédents . Cela permet donc d'ignorer l'historique des états précédents.
\par
Pour un état de Markov s et un état suivant s', l'état de transition est défini par : \[P_{ss'}=\mathbb{P}[S_{t+1}=s'|S_t=s]\]
Une matrice de transition P défini les probabilités de transitions d'un état s vers tout ses successeurs s' \[P = s
\begin{tabular}{c}
 s'\\
$$$\begin{bmatrix}
P_{11} & P_{12} & P_{13} \\
P_{21} & P_{22} & P_{23} \\
P_{31} & P_{32} & P_{33}
\end{bmatrix}$$$\\
\\
\end{tabular}\]
Ici $P_{12}$ est la probabilité d'aller vers l'état 2 quand on se trouve dans l'état 1. De ce fait la somme de chaque ligne de la matrice vaut 1.\par
Une chaine de Markov est une suite de variables aléatoires $S_1,S_2$... tel avec les propriété de Markov. Un processus de Markov(ou chaine de Markov) est un couple (S,P)
\begin{itemize}
\item[S] est un ensemble d'états supposé fini
\item[z] est une matrice de transitions 
\end{itemize}
\subsection{Processus de récompense de Markov}
Un Processus de récompense de Markov est une chaine de Markov avec des valeurs, défini par(S,P,R,$\gamma$) 
\begin{itemize}
\item[S] est un ensemble d'états supposé fini
\item[P] est une matrice de transitions 
\item[R] est la fonction de récompense $R_s = \mathbb{E}[R_{t+1}|S_t =s]$
\item[$\gamma$] est un facteur de réduction, $\gamma \in [0,1]$
\end{itemize}
\vspace{12pt}
Le résultat $G_t$ est le total des récompenses réduites depuis l'itération t.C'est une variable aléatoire.
\[G_t = R_{t+1}+\gamma R_{t+2}+...= \sum_{k=0}^{\infty}\gamma^k R_{t+k+1}\]
$\gamma$ sert a réduire la valeur de $R_k$ plus k est grand puisque on met $\gamma$ a la puissance k. Cela va mener, si  $\gamma$ est proche de 0, à une évaluation au court terme; et à une évaluation au long terme si  $\gamma$ est proche de 1. La réduction permet d'éviter les cycles dans les chaines de Markov  et permet de valoriser les récompenses au court terme 
La \textit{value function} $v(s)$ d'un processus de récompense de Markov est un fonction qui retourne la valeur au long terme en partant d'un état s. Cela représente l'espérance du retour de $G_t$.
\[v(s)=\mathbb{E}[G_t|S_t=s]\]
équation que nous pouvons réduire:
\[
\begin{aligned}
v(s)&=\mathbb{E}[G_t|S_t=s]\\
&=\mathbb{E}[R_{t+1}+\gamma R_{t+2}+\gamma^2R_{t+3}+...|S_t=s]\\
&=\mathbb{E}[R_{t+1}+\gamma(R_{t+2}+\gamma R_{t+3}+...)|S_t=s]\\
&=\mathbb{E}[R_{t+1}+\gamma G_t+1 |S_t=s]\\
&=\mathbb{E}[R_{t+1}+\gamma v(S_{t+1}) |S_t=s]
\end{aligned}
\]
La value function peut donc être divisé en deux parties, la récompense immédiate et la value function du successeur. Nous obtenons une équation de Bellman.
c'est une équation décrivant un problème de programmation dynamique c'est a dire que la valeur obtenu dépend du résultat de cette même fonction de façon récursive qui prend en compte l'état initial.
\\
cette équation peut être résumé en :
\[v(s)=R_s + \gamma \sum_{s'\in S} P_{ss'}v(s')\]
ce qui signifie que la \textit{value function} est égale a la récompense de l'état courant, plus la somme des values function des états suivant multiplié par leur probabilité d'être choisi, cette somme étant réduite par le facteur $\gamma$.\par
On peut résumer cette équation de Bellman en utilisant des vecteurs et matrices.
\[v=R+\gamma Pv\]
\begin{itemize}
\item[v] est le vecteur des value function
\item[R] est le vecteur des récompenses
\item[P] est la matrice de transition
\item[$\gamma$] est le facteur de réduction
\end{itemize}
on obtient donc une équation linéaire:
\[
\begin{aligned}
v &= R+ \gamma P v\\
(I - \gamma P)v &= R\\
v &= (I - \gamma P)^{-1} R
\end{aligned}
\]
Cette équation est en O($n^3$) pour n états ce qui n'est pas réalisable pour un nombre trop important d'états. on devra donc trouver des méthodes pour réduire cette complexité, comme la programmation dynamique,l'utilisation d'algorithmes de monte-Carlo ou l'apprentissage par différences temporelles.
\subsection{Le processus de décision de Markov}
Un processus de décision de Markov est un processus de récompense de Markov avec des décisions. Défini par (S,A,P,R,$\gamma$).
\begin{itemize}
\item[S] est un ensemble d'états supposé fini
\item[A] est un ensemble d'actions supposé fini
\item[P] est une matrice de transitions 
\item[R] est la fonction de récompense $R_s = \mathbb{E}[R_{t+1}|S_t =s]$
\item[$\gamma$] est un facteur de réduction, $\gamma \in [0,1]$
\end{itemize}
\vspace{12pt}
\subsubsection{Policies}
Une politique(\textit{policy}) $\pi$ est une distribution sur les action A sachant les états.
\[
\pi (a|s) = \mathbb{P}[A_t = a | S_t =s]
\]

Une politique définit le comportement d'un agent. Du fait de la propriété des chaines de Markov, les \textit{policies} ne dépendent pas de l'historique ni de l'itération.
\par
la séquence d'état et de récompenses avec une \textit{policy}, est un processus de récompense de Markov noté $(S,P^{\pi},R^{\pi},\gamma)$ ou:
\par
\[
P^{\pi }_{s,s'}= \sum_{a \in A} (a|s)P^{a}_{ss'}
\]
\[
R^{\pi}_{s}=\sum_{a \in A} (a|s)R^{a}_{s}
\]


\subsubsection{La value function}
La\textit{ value function} d'un état $v_{\pi}(s)$ d'un processus de décision de markov et le résultat attendu, en partant d'un état s, et suivant une policy $\pi$

\[
v_{\pi}(s) = \mathbb{E}_{\pi}[G_t|S_t = s]
\]

La fonction d' \textit{action value} $q_{\pi}(s,a)$ est est le retour attendu en partant de l'état s, réalisant l'action a et suivant la policy $\pi$.
\[
q_{\pi}(s,a) = \mathbb{E}_{\pi}[G_t|S_t =s,A_t =a]
\]

La valeur de $v^{\pi}$ attendu par l'équation de Bellman est :
\[
v_{\pi}(s)= \sum_{a \in A} \pi (a|s)q_{\pi}(s,a)
\]

La valeur de $q^{\pi}$ attendu par l'équation de Bellman est :

\[
q_{\pi}(s,a)= R^{a}_s + \gamma \sum_{s \in S} P^{a}_{ss'}v_{\pi}(s')
\]

En retournant sous forme matriciel on a:
\[
\begin{aligned}
v_{\pi} &= R^{\pi} + \gamma P^{\pi} v*_{pi}
v_{\pi} &= (I - \gamma P^{\pi})^{-1} R^{\pi}
\end{aligned}
\]
\subsubsection{value function optimal}

la value function d'un état $v_*(s)$ est le maximum de la value function de toutes les policies:
\[
 v_*(s) = \underset{\pi}{max}* v_{\pi}(s) 
\]
\[
 q_*(s,a) =\underset{\pi}{max}*q_{\pi}(s,a) 
\]
et
\[
  v_*(s) = \underset{a}{max}*R^{a}_s \sum_{s' \in S} P^{a}_{ss'} v_* (s') 
\]
\[
q_*(s,a) = R^{a}_s + \gamma \sum_{s' \in S} P^{a}_{ss'} \underset{a'}{max}* q_*(s',a') 
\]

pour résoudre ces équations il existe quelques solutions, comme par exemple le Q-learning, la \textit{policy iteration} et la \textit{value itaration}.

\section{Partie 2}

\section{Estimer la valeur d'un processus de décision de Markov inconnu}

\subsection{Les méthodes de Monte-Carlo}
	
Un algorithme de Monte-Carlo est un algorithme probabiliste. TODO:  def mc

Cette méthode apprend directement depuis une expérience. elle est aussi "model-free" TODO a traduire, c'est a dire qu'il n'a pas besoin de la connaissance de la chaine de Markov et de récompenses.
	

\end{document}